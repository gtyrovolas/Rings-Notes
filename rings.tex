\documentclass[11pt,a4paper]{article}
\usepackage[a4paper, total={6.5in, 8in}]{geometry}
\usepackage[utf8]{inputenc}
\usepackage{amsfonts}
\usepackage{amssymb}
\usepackage{amsmath}
\usepackage{mathtools}
\usepackage{amsthm}

\title{Rings}
\author{Giannis Tyrovolas}

\newtheorem{theorem}{Theorem}[section]

\theoremstyle{definition}
\newtheorem{definition}[theorem]{Definition}
\newtheorem{example}[theorem]{Example}
\newtheorem{corollary}[theorem]{Corollary}
\newtheorem{lemma}[theorem]{Lemma}

\DeclarePairedDelimiter\abs{\lvert}{\rvert}
\DeclarePairedDelimiter\norm{\lVert}{\rVert}

\begin{document}

\maketitle

\section{Introduction}

In the ``beautiful'' mental position that is Hilary of the second year I decided to study rings. What is a ring? Why is it important? We will hopefully learn. But quite possibly not. Let's end this section with a quote.\newline



\framebox[1.1\width]{If you liked it then you shoulda put a ring on it - Beyonce} \par

\section{Recap on rings}

Let's start this section with the obvious thing we should define. The ring!

\begin{definition}[Ring]

A ring $(R, +, \times)$ is made of a set $R$ (also called the carrier set) and two binary operations $+: R \times R \longrightarrow R$, $\times: R \times R \longrightarrow R$. Such that:
\begin{enumerate}
	\item $(R, +)$ is an abelian group.
	\item $\times$ is associative 
	\item $\times$ distributes over $+$ i.e. $\forall a,b,c \in R$\newline
$$
	a \times (b + c) = a \times b + a \times c \text{ and} (a + b) \times c = a \times c + b \times c
$$
\end{enumerate}

If $R$ has a multiplicative identity we call $R$ \emph{unital}
\end{definition}

\begin{theorem}[Basic properties]
\begin{enumerate}
	\item Zero is annihilating: $\forall r \in R \; 0_R r = 0_R = r 0_R$
	\item $(-x)y = -xy = x(-y)$
\end{enumerate}
\end{theorem}

\begin{proof}
\begin{enumerate}
	\item $0_R r = (0_R + 0_R) r = 0_R r + 0_R r \implies 0_R r = 0_R$
	\item $xy + (-x)y = (x - x)y = 0_R y = 0_R$ Hence by the uniqueness of the inverse of $-xy$, $(-x)y = -xy$. Similarly for $x(-y)$
\end{enumerate}
\end{proof}

\begin{definition}[Unit group]
For a \emph{unital} ring $R$ denote $U(R)$ the set of $r \in R$ with a multiplicative inverse. 
\end{definition}

\begin{theorem}
$(U(R), \times)$ is a group.  
\end{theorem}

\begin{proof}
\begin{enumerate}
	\item Associativity is inherited
	\item Identity: $1_R \in U(R)$ trivially.
	\item Inverse: By definition of $U(R)$
	\item Closure: Let $x, y \in U(R)$ then $\exists x^{-1}, y^{-1} \in U(R)$. \newline Then $(xy)(y^{-1}x^{-1}) = 1_R \implies xy \in U(R)$
\end{enumerate}
\end{proof}

\begin{definition}[Subrings]
Let $R$ be a ring and $S \subseteq R$ such that $S$ is a ring. Then $S$ is a subring.
If $R$ is a unital ring and $1_R \in S$ then $S$ is a unital subring.
\end{definition}

\begin{lemma}[Closure of intersection]
Let $\mathcal{Q}$ a set of subrings of a ring $R$. Then $\bigcap\limits_{S \in \mathcal{Q}} S$ is a subring of $R$.
\end{lemma}

\begin{definition}[Generated Subring]
\[
	S[\lambda_1,..., \lambda_n] = \bigcap\{T : T \text{ is a subring of }R \text{ and } \lambda_1,\ldots, \lambda_n \in T \text{ and } S \subseteq T \}
\]
\end{definition}

\begin{definition}[Homomorphisms]
Let $\phi: R \longrightarrow S$ where $R, S$ rings. And $\forall x, y \in R$:
\[
	\phi(x + y) = \phi(x) + \phi(y) \ \ \ \ \ \text{and} \ \ \ \ \ \  \phi(xy) = \phi(x) \phi(y)
\]
Then $\phi$ is a homomorphism.

If $R, S$ are unital and $\phi(1_R) = 1_S$ then $\phi$ is a unital homomorphism.
\end{definition}

\begin{lemma}[Inverses]
Let $\phi: R \longrightarrow S$ a homomorphism. Then $\phi(0_R) = 0_S$ and $\forall r \in R$,  $\phi(-r) = -\phi(r)$. If $\phi$ is unital then $\forall x \in U(R)$, $\phi(x) \in U(S)$ and $\phi(x)^{-1} = \phi(x^{-1})$.
\end{lemma}

\begin{proof}
\[
	\phi(0_R) = \phi(0_R + 0_R) = \phi(0_R) + \phi(0_R) \implies \phi(0_R) = 0_S
\]
\[
	0_S = \phi(0_R) = \phi(x) + \phi(-x) \implies \phi(-x) = -\phi(x) 
\]
Similarly for $\phi(x)^{-1} = \phi(x^{-1})$
\end{proof}

\section{Integral Domains and Polynomials}

\begin{definition}[Zero dividers]
$r \in R$ is a (left) \emph{zero divider} if $\exists s \in R^*$ such that $rs = 0_R$
\end{definition}
% add the examples

\begin{definition}[Integral Domains]
$R$ is an integral domain if it is a non-trivial, commutative, unital ring with no zero-divisors.
\end{definition}

\begin{lemma}
	If $R$ is an integral domain then if $x\in R^*$ $xy=xz \implies y = z$
\end{lemma}

\begin{proof}
\[
	0_R = xy - xz = x(y-z) \implies y-z = 0\implies y = z \text{ since } x \text{ is not a zero-divider}
\]
\end{proof}

This proof is refered to the notes as ``cute''. Its a 7 at best.

\begin{theorem} 
A finite integral domain is a field.
\end{theorem}

\begin{proof}
Consider $R \longrightarrow R$, $x \mapsto ax$ where $a \in R^*$. This map is injective by cancellation. Since $R$ is finite it is also surjective and hence $\exists r \in R$ such that $ar = 1_R$. By commutativity it is a two sided inverse. So $a$ has an inverse and since $a$ is arbitrary $R$ is a field.
\end{proof}

Basic stuff about polynomials, what you'd expect without having done any rings.

\begin{definition}[Polynomial]
Let $R$ be a non-trivial commutative, unital ring. Then we write $R[X]$ the set of $R$-polynomials with coefficients in $R$ and variable $X$. These are of the form:
\[
	p(X) = \sum_{i=0}^\infty r_i X^i
\]
where $r_i \in R$ and $r_i \in R^*$ for finitely many $i$.

Two polynomials are equal if all their coefficients are equal. Also let polynomials $p, q$ with coefficients $a_i$, $b_i$. Then:

\[
	(p + q)(X)= \sum_{i = 0}^\infty(a_i + b_i)X^i \ \ \ (pq)(X) = \sum_{i = 0}^\infty(\sum_{j=0}^{i}a_jb_{i-j})X^i 	  	
\]  
\end{definition}

\begin{theorem}
The following are equivalent:
\begin{enumerate}
	\item $R$ is an integral domain
	\item $R[X]$ is an integral domain
	\item $p, q \in R[X]^* \implies pq \in R[X]^*$ and $\deg pq = \deg p + \deg q$ 
	\item A polynomial of degree $d$ has at most $d$ roots
\end{enumerate}
\end{theorem}

\begin{proof}
Most of these are trivial. (2) implies (1) by considering constant polynomials. (3) implies (2) by definition. Now, (1) implies (3) since for $\deg p = n$, $\deg q = m$ the coefficient $n+m$th coefficient of $pq$ is $a_n b_m$. Since $R$ is an integral domain and $a_n, b_m \neq 0_R$, $a_n b_m \neq 0_R \implies \deg pq = n + m$

For (4) implying (1): Consider the polynomial $p(X) = rX$, $r \neq 0_R$. Since $p(X)$ has only one root and $0_R$ is a root there are no other roots. So $R$ is an integral domain.

Now for (1) + (2) $\implies$ (4). 
\end{proof}


\end{document}

